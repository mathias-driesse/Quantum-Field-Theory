\documentclass{article}
\usepackage{physics, amsmath, amsfonts, empheq, mathtools, booktabs, hyperref}
\usepackage[style=nature, autocite = inline]{biblatex}
\usepackage[margin=1in]{geometry}

\setlength{\parindent}{0pt}

\newcommand*\widefbox[1]{\fbox{\hspace{2em}#1\hspace{2em}}}
\newcommand{\normord}[1]{:\mathrel{#1}:}

\title{Canonical Quantum Field Theory}
\author{Mathias Driesse}

\begin{document}

\maketitle

\section{Canonical Quantization of a Scalar Field}

A scalar field $\phi(x)$ that obeys the Klein-Gordon equation
\begin{align}
    (\partial^2 + m^2)\phi(x) = 0
\end{align}

also obeys the following \textit{equal-time commutation relations} (ETCRs):
\begin{gather}
    [\phi(t, \underline{x}), \dot{\phi}(t, \underline{x}^\prime)] = i\delta(\underline{x} - \underline{x}^\prime) \\
    [\phi(t, \underline{x}), \phi(t, \underline{x}^\prime)] = [\dot{\phi}(t, \underline{x}), \dot{\phi}(t, \underline{x}^\prime)] = 0
\end{gather}

A scalar field may be expanded in terms of mode operators:
\begin{align}
    \phi(t, \underline{x}) &= \phi^+(x) + \phi^-(x) \\
    &= \int \frac{d^3 p}{(2 \pi)^3} \frac{1}{2 \omega(\underline{p})}\left(a(\underline{p}) e^{-i p \cdot x}+a(\underline{p})^{\dagger} e^{i p \cdot x}\right)
\end{align}

where the creation and annihilation operators in momentum space obey the commutation relations
\begin{align}
    [a (\underline{p}), a^\dagger(\underline{p}^\prime)] = 2\omega(\underline{p}) (2\pi)^3 \delta(\underline{p}-\underline{p}^\prime) \\
    [a (\underline{p}), a(\underline{p}^\prime)] = [a^\dagger(\underline{p}), a^\dagger(\underline{p}^\prime)] = 0.
\end{align}

To prevent infinite contributions to the energy, we require that all terms be \textit{normal ordered} - that is, all annihilation operators stand to the right of all creation operators. This results in the convenient property
\begin{empheq}[box=\widefbox]{align*}
    \ev{\normord{\phi(x)\phi(y)}}{0} = 0.
\end{empheq}

We may simplify covariant commutation relations for the Klein-Gordon field by introducing three invariant functions
\begin{gather}
    \Delta^+(x) \equiv -i \int \dfrac{d^3p}{(2\pi)^3} \dfrac{1}{2\omega(\underline{p})}e^{-ip \cdot x} \\
    \Delta^-(x) \equiv \Delta^+(x)^* = i \int \dfrac{d^3p}{(2\pi)^3} \dfrac{1}{2\omega(\underline{p})}e^{ip \cdot x}\\
    \Delta(x) \equiv \Delta^+(x) + \Delta^-(x).
\end{gather}

All three satisfy the Klein-Gordon equation, and we may furthermore write $\Delta$ in a manifestly covariant form:
\begin{equation}
    \Delta(x) = -i \int \hat{d}^4p \hat{\delta}(p^2-m^2)\theta(p_0)e^{-ip\cdot x}.
\end{equation}

$\Delta$ also satisfies the microcausality condition
\begin{empheq}[box=\widefbox]{align*}
    [\phi(x), \phi(y)] = 0, \quad (x-y)^2 < 0.
\end{empheq}

\section{Interactions and Scattering in $\phi^3$ Theory}

The Lagrangian for this theory is given by 
\begin{align}
    \mathcal{L} &\equiv \mathcal{L}_0 + \mathcal{L}_I \\
    &= \frac{1}{2} \partial_\mu \phi \partial^\mu \phi - \frac{1}{2}m^2\phi^2 - \frac{1}{3!} g \phi^3.
\end{align}

We may write the Euler-Lagrange equation as 
\begin{align}
    (\partial^2 + m^2)\phi = -\frac{1}{2}g\phi^2.
\end{align}

Since this is a nonlinear partial differential equation, we would like to proceed by solving it in terms of powers of coupling $g$.

We define the S-matrix as the time evolution operator between two states in the far past and the far future at which the particles involved may be considered non-interacting:
\begin{align}
    \ket{\Psi, \infty} = S\ket{\Psi, -\infty}.
\end{align}

Since in general, we must consider many possible initial and final states, we consider the $S$-matrix elements defined as
\begin{align}
    S_{fi} = \mel{f}{S}{i}
\end{align}

Note that the $S$-matrix is unitary ($S^\dagger S = SS^\dagger = 1$) even when particles in the state are destroyed and/or created.

Furthermore, in the future we will work in the interaction picture of quantum mechanics - a mix of the Schrödinger and Heisenberg pictures. We split the Hamiltonian into free and non-interacting parts $H = H_0 + H_I$ and let the time evolution of the state be governed by $H_I$, and the time evolution of the operators by $H_0$:
\begin{gather}
    \frac{\partial}{\partial t} O(t) = i [H_0, O(t)] \\
    i \frac{\partial}{\partial t} \ket{\psi, t} = H_I \ket{\psi, t}. \label{eq:schrodinger}
\end{gather}

Integrating Eq. \ref{eq:schrodinger}, we obtain a recursion relation for $\ket{\Psi, t}$:
\begin{align}
    \ket{\Psi, t} &= \ket{\Psi, -\infty} + \int_{-\infty}^t dt_1 \frac{\partial}{\partial t}\ket{\Psi, t_1} \\
    &= \ket{i} - i \int_{-\infty}^t dt_1 H_I(t_1) \ket{\Psi, t_1}
\end{align}

\end{document}