%!TEX program = lualatex
\documentclass{article}
\usepackage{physics, amsmath, amsfonts, empheq, mathtools, booktabs, hyperref, simpler-wick, tikz-feynman, slashed}
\usepackage[style=nature, autocite = inline]{biblatex}
\usepackage[margin=1in]{geometry}

\setlength{\parindent}{0pt}

\newcommand*\widefbox[1]{\fbox{\hspace{2em}#1\hspace{2em}}}
\newcommand{\normord}[1]{:\mathrel{#1}:}

\title{Canonical Quantum Field Theory}
\author{Mathias Driesse}

\begin{document}

\maketitle

\section{Canonical Quantization of a Scalar Field}

A scalar field $\phi(x)$ that obeys the Klein-Gordon equation
\begin{align}
    (\partial^2 + m^2)\phi(x) = 0
\end{align}

also obeys the following \textit{equal-time commutation relations} (ETCRs):
\begin{gather}
    [\phi(t,\underline{x}),\pi(t,\underline{x}')] = [\phi(t, \underline{x}), \dot{\phi}(t, \underline{x}^\prime)] = i\delta(\underline{x} - \underline{x}^\prime) \\
    [\phi(t, \underline{x}), \phi(t, \underline{x}^\prime)] = [\pi(t,\underline{x}), \pi(t,\underline{x}')]=[\dot{\phi}(t, \underline{x}), \dot{\phi}(t, \underline{x}^\prime)] = 0
\end{gather}

where the conjugate momentum $\pi(x)$ is the usual $\pdv{\mathcal{L}}{\dot{\phi}(x)}$. A scalar field may be expanded in terms of mode operators:
\begin{align}
    \phi(t, \underline{x}) &= \phi^+(x) + \phi^-(x) \\
    &= \int \frac{d^3 p}{(2 \pi)^3} \frac{1}{2 \omega(\underline{p})}\left(a(\underline{p}) e^{-i p \cdot x}+a(\underline{p})^{\dagger} e^{i p \cdot x}\right)
\end{align}

where the creation and annihilation operators in momentum space obey the commutation relations
\begin{align}
    [a (\underline{p}), a^\dagger(\underline{p}^\prime)] = 2\omega(\underline{p}) (2\pi)^3 \delta(\underline{p}-\underline{p}^\prime) \\
    [a (\underline{p}), a(\underline{p}^\prime)] = [a^\dagger(\underline{p}), a^\dagger(\underline{p}^\prime)] = 0.
\end{align}

To prevent infinite contributions to the energy, we require that all terms be \textit{normal ordered} - that is, all annihilation operators stand to the right of all creation operators. This results in the convenient property
\begin{empheq}[box=\widefbox]{align*}
    \ev{\normord{\phi(x)\phi(y)}}{0} = 0.
\end{empheq}

We may simplify covariant commutation relations for the Klein-Gordon field by introducing three invariant functions
\begin{gather}
    \Delta^+(x) \equiv -i \int \dfrac{d^3p}{(2\pi)^3} \dfrac{1}{2\omega(\underline{p})}e^{-ip \cdot x} \\
    \Delta^-(x) \equiv \Delta^+(x)^* = i \int \dfrac{d^3p}{(2\pi)^3} \dfrac{1}{2\omega(\underline{p})}e^{ip \cdot x}\\
    \Delta(x) \equiv \Delta^+(x) + \Delta^-(x).
\end{gather}

All three satisfy the Klein-Gordon equation, and we may furthermore write $\Delta$ in a manifestly covariant form:
\begin{equation}
    \Delta(x) = -i \int \hat{d}^4p \hat{\delta}(p^2-m^2)\theta(p_0)e^{-ip\cdot x}.
\end{equation}

$\Delta$ also satisfies the microcausality condition
\begin{empheq}[box=\widefbox]{align*}
    [\phi(x), \phi(y)] = 0, \quad (x-y)^2 < 0.
\end{empheq}

\section{Interactions and Scattering in $\phi^3$ Theory}

The Lagrangian for this theory is given by 
\begin{align}
    \mathcal{L} &\equiv \mathcal{L}_0 + \mathcal{L}_I \\
    &= \frac{1}{2} \partial_\mu \phi \partial^\mu \phi - \frac{1}{2}m^2\phi^2 - \frac{1}{3!} g \phi^3.
\end{align}

We may write the Euler-Lagrange equation as 
\begin{align}
    (\partial^2 + m^2)\phi = -\frac{1}{2}g\phi^2.
\end{align}

Since this is a nonlinear partial differential equation, we would like to proceed by solving it in terms of powers of coupling $g$.

We define the S-matrix as the time evolution operator between two states in the far past and the far future at which the particles involved may be considered non-interacting:
\begin{align}
    \ket{\Psi, \infty} = S\ket{\Psi, -\infty}.
\end{align}

Since in general, we must consider many possible initial and final states, we consider the $S$-matrix elements defined as
\begin{align}
    S_{fi} = \mel{f}{S}{i}
\end{align}

Note that the $S$-matrix is unitary ($S^\dagger S = SS^\dagger = 1$) even when particles in the state are destroyed and/or created.

Furthermore, in the future we will work in the interaction picture of quantum mechanics - a mix of the Schrödinger and Heisenberg pictures. We split the Hamiltonian into free and non-interacting parts $H = H_0 + H_I$ and let the time evolution of the state be governed by $H_I$, and the time evolution of the operators by $H_0$:
\begin{gather}
    \frac{\partial}{\partial t} O(t) = i [H_0, O(t)] \\
    i \frac{\partial}{\partial t} \ket{\psi, t} = H_I \ket{\psi, t}. \label{eq:schrodinger}
\end{gather}

Integrating Eq. \ref{eq:schrodinger}, we obtain a recursion relation for $\ket{\Psi, t}$:
\begin{align}
    \ket{\Psi, t} &= \ket{\Psi, -\infty} + \int_{-\infty}^t dt_1 \frac{\partial}{\partial t}\ket{\Psi, t_1} \\
    &= \ket{i} - i \int_{-\infty}^t dt_1 H_I(t_1) \ket{\Psi, t_1}.
\end{align}

By substituting the recursion relation into itself, we write the Dyson series explicitly as
\begin{align}
    S=1+\sum_{n=1}^{\infty}(-i)^n \int_{-\infty}^{\infty} d t_1 \int_{-\infty}^{t_1} d t_2 \cdots \int_{-\infty}^{t_{n-1}} d t_n H_I\left(t_1\right) H_I\left(t_2\right) \cdots H_I\left(t_n\right).
\end{align}

The introduction of the time ordered product of two operators
\begin{align}
    T(A(t_1)B(t_2)) = \theta(t_1 - t_2)A(t_1)B(t_2) + \theta(t_2 - t_1)B(t_2)A(t_1)
\end{align}

and the use of the Hamiltonian density instead of the Hamiltonain yields the manifestly covariant expression

\begin{empheq}[box=\widefbox]{align*}
    S=\sum_{n=0}^{\infty} \frac{(-i)^n}{n !} \int d^4 x_1 \int d^4 x_2 \cdots \int d^4 x_n T\left(\mathcal{H}_I\left(x_1\right) \mathcal{H}_I\left(x_2\right) \cdots \mathcal{H}_I\left(x_n\right)\right).
\end{empheq}

Since our calculations involve time-ordered products through \textit{Wick's Theorem}, which we will see shortly, it is convenient to define the Feynman propagator
\begin{align}
    \Delta_F(x-x') \equiv \theta(t)\Delta^+(x) - \theta(-t)\Delta^-(x).
\end{align}

Thus, making use of our knowledge of the functions $\Delta^{\pm}$, we find that
\begin{empheq}[box=\widefbox]{align*}
    i\Delta_F(x-x') = \wick{\c\phi (x) \c\phi (x')} = \ev{T\left(\phi(x) \phi(x')\right)}{0},
\end{empheq}

where the bracket on top denotes a \textit{Wick contraction}. How can we relate the time-ordered product in general, normal ordering and Wick contractions? The answer is \textit{Wick's Theorem}. For unequal times
\begin{equation}
\begin{aligned}
    T(ABCD\cdots WXYZ) &= \, \normord{ABCD \cdots WXYZ} \\
        & + \normord{\wick{\c A \c B}C\cdots YZ} + \normord{\wick{\c1 A B \c1 C} \cdots YZ} + \cdots + \normord{ABC \cdots \wick{\c Y \c Z}} \\
        & + \normord{\wick{\c1 A \c1 B \c1 C \c1 D} \cdots WXYZ} + \cdots + \normord{ABCD \cdots \wick{\c1 W \c1 X \c1 Y \c1 Z}} \\
        & + \cdots,
\end{aligned}
\end{equation}

which simplifies to, for example, in the case of a single time ordering to 
\begin{equation}
    T(A(x)B(x')) = \normord{A(x)B(x')} + \ev{T(A(x)B(x'))}{0}.
\end{equation}

When performing the Dyson expansion, we will need to evaluate the Feynman propagator many times. Therefore, it is usefeful to rewrite it in terms of a single integral

\begin{equation}
    \Delta_F(x) = \int \hat{d}^4p \frac{e^{-ip\cdot x}}{p^2-m^2+i\varepsilon}
\end{equation}

where $\varepsilon \equiv 2\eta \omega(\underline{p})$ is a small number that takes care of the contours in the complex plane that we have to take due to the poles at $p=\pm m$. In fact, $\Delta_F$ is a Green's function for the Klein-Gordon equation, corresponding to Dirichlet boundary conditions (initial and final conditions on $\phi(x)$), and is thus appropriate to our quantum paradigm of initial and final states.

As an example, let us expand the $S$-matrix in position space. In $\phi^3$ theory, we have $\mathcal{H}_I(x) = \frac{g}{3!}\normord{\phi^3(x)}$, which we expand in terms of $\phi^+$ (which destroys a particle at $x$) and $\phi^-$ (which creates a particle at $x$).

\begin{enumerate}
    \item For $n=0$, nothing happens.
    \item For $n=1$,
    \begin{equation}
        S^{(1)} =\frac{-ig}{3!} \int d^4x \, T(\normord{\phi^3(x)}).
    \end{equation}
        Since there are no \textit{unequal times}, there are no terms with contractions. The terms without contractions cannot conserve momentum, so they vanish.
    \item For $n=2$,
    \begin{equation}
        \begin{aligned}
            S^{(2)} = -\frac{g^2}{2!(3!)^2} \int d^4x d^4y \, T\left(\normord{\phi^3(x)}\normord{\phi^3(y)}\right),
        \end{aligned}
    \end{equation}
        which we may split up into terms containing zero, one, two and three contractions.
    \begin{enumerate}
        \item $S_0^{(2)}$ only contains one term, and it is proportional to $\normord{\phi^3(x)\phi^3(y)}$. The interactions at $x$ and $y$ are unrelated, so they factorize and vanish due to momentum conservation as in the $n=1$ case.
        \item $S_1^{(2)}$ contains terms with one contraction, i.e. proportional to $\wick{\c\phi(x) \c\phi(y)} \normord{\phi^2(x) \phi^2(y)}$. Expanding it out in terms of $\phi^+$ and $\phi^-$, there are three terms that survive momentum conservation:
        \begin{equation} \label{eq:s12}
            \begin{aligned}
                \phi^-(y)\phi^-(y)\wick{\c\phi(x) \c\phi(y)}\phi^+(x)\phi^+(x), \\
                \phi^-(x)\phi^-(y)\wick{\c\phi(x) \c\phi(y)}\phi^+(x)\phi^+(y), \\
                \phi^-(y)\phi^-(x)\wick{\c\phi(x) \c\phi(y)}\phi^+(x)\phi^+(y).
            \end{aligned}
        \end{equation}
        % which correspond to the three Feynman diagrams
        % \begin{figure}[!h]
        %     \feynmandiagram [horizontal=a to b]{
        %         i1 -- [scalar] a -- [scalar] i2,
        %         a -- [scalar] b,
        %         f1 -- [scalar] b -- [scalar] f2,
        %     };
        %     \feynmandiagram [vertical=a to b]{
        %         i1 -- [scalar] a -- [scalar] f1,
        %         a -- [scalar] b,
        %         i2 -- [scalar] b -- [scalar] f2,
        %     };
        % \end{figure}
        There is also a combinatorial $3^2$ factor, since there are 3 ways of contracting $\phi(x)$ and 3 ways of contracting $\phi(y)$.
        \item $S_2^{(2)}$ is given by the term $\wick{\c1\phi(x)\c1\phi(y)\c1\phi(x)\c1\phi(y)} \normord{\phi(x)\phi(y)}$, which diverges and forms the so-called \textit{self-energy correction}.
        \item $S_3^{(2)}$ is given by the term $\wick{\c1\phi(x)\c1\phi(y)\c1\phi(x)\c1\phi(y)\c1\phi(x)\c1\phi(y)}$, which also diverges, but as it is disconnected it makes no contribution to the scattering process.
    \end{enumerate}
\end{enumerate}

This is a lot of work for calculating the $S$-matrix! Fortunately, we are more interested in calculating the matrix elements $S_{fi} = \mel{f}{S}{i}$ than $S$ itself. Since we label free particle initial and final states by their definite on-shell momentum $\ket{\underline{p}} = a^\dagger(\underline{p})\ket{0}$, we should work in momentum space. Then
\begin{equation} \label{eq:phiplusp}
\begin{aligned}
    \phi^+(x)\ket{\underline{p}} &= \int  \frac{\hat{d}^3 \underline{p}}{2E'} e^{-ip'\cdot x} a(\underline{p}')a^\dagger(\underline{p})\ket{0} \\
    &= \int d^3 \underline{p}'   e^{-ip'\cdot x} \delta^3(\underline{p}-\underline{p}')\ket{0} \\
    &= e^{-ip\cdot x}\ket{0}
\end{aligned}
\end{equation}

while
\begin{equation} \label{eq:phiminusp}
\begin{aligned}
    \bra{\underline{p}}\phi^-(x) &= \bra{0} a(\underline{p}) \int \frac{\hat{d}^3 \underline{p}}{2E'} e^{ip'\cdot x} a^\dagger(\underline{p}) \\
    &= \bra{0} \int d^3 \underline{p}' e^{ip'\cdot x} \delta^3(\underline{p}-\underline{p}') \\
    &= e^{ip\cdot x} \bra{0}.
\end{aligned}
\end{equation}

As an example, let us perform a momentum-space calculation of the 2$\rightarrow$2 process through the $s$-channel. We may do this by simply inserting the first line of Eq. \ref{eq:s12}:
\begin{equation}
\begin{aligned}
    \mel{f}{S_{1,s}^{(2)}}{i} &= -\frac{3^2 g^2}{(3!)^2} \int d^4x \int d^4y \mel{q;q'}{\phi^-(y)\phi^-(y)\wick{\c\phi(x) \c\phi(y)}\phi^+(x)\phi^+(x)}{p;p'} \\
    &= -\frac{g^2}{4} \int d^4x \int d^4y \, 4 e^{i(q+q')\cdot x} \, i \int \hat{d}^4k \frac{e^{-ik\cdot (y-x)}}{k^2-m^2+i\varepsilon} e^{-i(p+p')\cdot x} \\
    &= -ig^2 \int \hat{d}^4k \,\hat{\delta}^4\left(q+q'-k\right)\hat{\delta}^4\left(k-p-p'\right) \frac{1}{k^2-m^2+i\varepsilon} \\
    &= \hat{\delta}^4\left(q+q'-p-p'\right)\frac{-ig^2}{(p+p')^2-m^2},
\end{aligned}
\end{equation}

where on the second line we have applied Eqs. \ref*{eq:phiplusp}, \ref*{eq:phiminusp} and inserted our expression for the Feynman propagator, with an additional factor of 4 coming from the choice of apply $(\phi^+)^2$ on $\ket{p;p'}$. For the other two contributions we have the same, replacing the denominator by $(p-q)^2-m^2$ and $(p-q')-m^2$ respectively. It is therefore useful to write down the \textit{Mandelstam invariants} $s$, $t$, $u$, defined as 
\begin{equation}
\begin{aligned}
    s=(p_1+p_2)^2=(p_3+p_4)^2 \\
    t=(p_1-p_3)^2=(p_2-p_4)^2 \\
    u=(p_1-p_4)^2=(p_2-p_3)^2.
\end{aligned}
\end{equation}

Applying a similar process to the loop diagram leads to a divergent term, which we will look at in more detail through $renormalization$. 

\subsection{Feynman rules for scalar field theory}

An attentive reader, having calculated all three channels, might have noticed that the process above seems fairly simple and algorithmic. Indeed, we can define the so-called \textit{Feynman rules}, which for any theory allow us to write down the contribution from each diagram without performing any nasty integrations. For scalar $\phi^3$ theory, we have:

\begin{itemize}
    \item for each $\phi^3$ vertex, a factor of $-ig$ and a delta function ensuring momentum conservation
    \item for each external line, a factor of $1$
    \item for each internal line of momentum $k$, a factor $i/(k^2-m^2+i\varepsilon)$
    \item for each momentum $k$ not fixed by momentum conservation, an integral $\int \hat{d}^4k$
    \item a symmetry factor.
\end{itemize}

\section{Complex Scalar Fields and Charge Conservation}

In quantum field theory, for the $S$-matrix to be unitary, we only require that the Lagrangian density $\mathcal{L}$, the Hamiltonian density $\mathcal{H}$, and the action $S$ be real. Thus, a quantum field may also be complex as long as above conditions are satisfied. They also have the important new property of charge conservation.

If we have a multiplet (several) of scalar fields $\phi_r(x)$, where $r=1,2,\cdots N$, then we may quantize them as we have done for a single scalar field, the only difference being an additional delta function in the nonzero commutators.

A complex scalar field $\phi(x)$ can be expressed in terms of two independent fields $\phi_1$ and $\phi_2$:
\begin{equation}
    \phi(x)=\frac{1}{\sqrt{2}}(\phi_1(x) + i\phi_2(x)).
\end{equation}

Then we have the Lagrangian density
\begin{equation}
    \mathcal{L} = \left(\partial_\mu\phi^\dagger(x)\right)\left(\partial^\mu\phi(x)\right)-m^2\phi^\dagger(x)\phi(x)
\end{equation}

along with the conjugate momentum $\pi(x)=\pdv{\mathcal{L}{\dot{\phi}(x)}}=\dot{\phi}^\dagger(x)$, which provides the ETCRs
\begin{equation}
    [\phi(t,\underline{x}), \pi(t,\underline{x}')]=[\phi^\dagger(t,\underline{x}),\pi^\dagger(t,\underline{x}')] = i\delta(\underline{x}-\underline{x}').
\end{equation}

Each of the fields $\phi(x)$ and $\phi^\dagger(x)$ obeys the Klein-Gordon equation independently. We expand the fields in terms of mode operators
\begin{equation}
\begin{aligned}
    \phi(x) = \int \hat{d}^3p \frac{1}{2\omega(\underline{p})} \left(a(\underline{p})e^{-ip\cdot x} + b^\dagger(\underline{p})e^{ip\cdot x}\right),
\end{aligned}
\end{equation}

noting that we can no longer assume the Hermiticity relation between the operator coefficients of the positive and negative frequency terms because $\phi(x)$ is not Hermitian. Furthermore, the commutation relations 
\begin{equation}
\begin{aligned}
    [a(\underline{p}), a^\dagger(\underline{p}')] = [b(\underline{p}), b^\dagger(\underline{p'})] = 2\omega(\underline{p}) \hat{\delta}^3(\underline{p}-\underline{p}')
\end{aligned}
\end{equation}

with all others being zero reproduce the ETCRs. $a$ and $b$ and their Hermitian conjugates are the creation and annihilation operators for two types of particles. Furthermore, momentum is conserved, with the 4-momentum operator (not to be confused with the conjugate of the field) being 
\begin{gather}
    P^\nu = \int d^3x T^{0\nu}(x)\\
    T^{\mu\nu}(x) = \normord{\partial^\mu \phi^\dagger(x) \partial^\nu \phi(x) + \partial^\nu \phi^\dagger(x) \partial^\mu \phi(x) - \eta^{\mu\nu} \mathcal{L}(x)},
\end{gather}

or in terms of mode operators
\begin{equation}
    P^\nu = \int \hat{d}^3p \frac{1}{2\omega(\underline{p})}p^\nu \left(a^\dagger(\underline{p}) a(\underline{p}) + b^\dagger(\underline{p})b(\underline{p})\right).
\end{equation}

\subsection{Charge conservation}

Additionally, notice that the Lagrangian density is invariant under the phase transformations

\begin{equation}
\begin{aligned}
    \phi(x) \rightarrow \phi'(x)&=e^{i\alpha}\phi(x) \approx (1+i\alpha)\phi(x) \\
    \phi(x) \rightarrow \phi'(x)&=e^{-i\alpha}\phi(x) \approx (1-i\alpha)\phi(x),
\end{aligned}
\end{equation}

which is known as a \textit{global phase transformation} or \textit{gauge transformation of the first kind}. Furthermore, since the Lagrangian density itself is invariant, according to Noether's theorem there is a conserved current $j$ and conserved charge $Q$:
\begin{gather}
    Q = \int d^3x j^0(x) \\
    j^\mu(x) = -\frac{1}{\alpha}\left(\pdv{\mathcal{L}}{\partial_\mu\phi} + \pdv{\mathcal{L}}{\partial_\mu\phi^\dagger}\right) = i \normord{\phi^\dagger(x)\left(\partial^\mu \phi(x)\right) - \left(\partial^\mu \phi^\dagger(x)\right)\phi(x)}.
\end{gather}

Thus, the existence of charge is directly tied to the existence of complex scalar fields. Furthermore, we can think of $a$ as corresponding to particles with ``charge'' $+1$ while $b$ would correspond to particles with ``charge'' $-1$. Since $Q$ is conserved, these particles must always be created or destroyed in pairs - they are anti-particles of each other.

\subsection{Feynman rules for complex scalar fields}

The propagator for complex scalar fields is the same as for real scalar fields:
\begin{equation}
    \wick{\c\phi(x)\c\phi^\dagger(x')} = i\Delta_F(x-x').
\end{equation}

The external states are given by 
\begin{equation}
    \ket{p,+} = a^\dagger(\underline{p})\ket{0}, \quad \ket{p,-}=b^\dagger(\underline{p})\ket{0},
\end{equation}

so then
\begin{gather}
    \phi_+(x)\ket{p,+} = e^{-ip\cdot x}\ket{0}, \quad \phi_+^\dagger(x)\ket{p,-} = e^-ip\cdot x \ket{0} \\
    \bra{p,+}\phi_-^\dagger(x)=e^{ip\cdot x} \bra{0}, \quad \bra{p,-} \phi_-(x) = e^{ip\cdot x} \bra{0}.
\end{gather}

In terms of the interactions possible, remembering that the Lagrangian must remain real, we commonly use
\begin{equation}
    \mathcal{L}_I^4 = -\frac{1}{4}\lambda (\phi^\dagger\phi)^2
\end{equation}
or if there is also a real scalar field $\Phi$ 
\begin{equation}
    \mathcal{L}_I^3 = -y\Phi(\phi^\dagger\phi).
\end{equation}

These additional processes allow much more interesting processes such as elastic scattering, decay, and charged particle elastic scattering. 

\subsection{$C$, $P$, and $T$ for scalar fields}

Under \textbf{parity}, $x^\mu=(t,\underline{x}) \rightarrow \bar{x}^\mu = (t, -\underline{x})$, $p^\mu = (E, \underline{p}) \rightarrow \bar{p}^\mu = (E, -\underline{p})$. This is implemented using the unitary operator $\mathcal{P}$, thus
\begin{equation}
    \mathcal{P} \ket{0}, \quad \mathcal{P} \ket{\underline{p}} = \ket{-\underline{p}}.
\end{equation}

Since $\ket{\underline{p}} = a^\dagger(\underline{p})\ket{0}$, we must have
\begin{equation}
    \mathcal{P}a^\dagger(\underline{p})\mathcal{P}^\dagger = a^\dagger(-\underline{p}), \quad \mathcal{P}a(\underline{p})\mathcal{P}^\dagger = a(-\underline{p}).
\end{equation}

Therefore
\begin{equation}
    \phi(x) \rightarrow \mathcal{P} \phi(x) \mathcal{P}^\dagger = \phi(\bar{x}).
\end{equation}

Under \textbf{time reversal}, $x^\mu \rightarrow -\bar{x}^\mu$, $p^\mu \rightarrow \bar{p}^\mu$. Furthermore, initial and final states are interchanged $\mel{f}{\,}{i} \rightarrow \mel{i}{\,}{f}$, or alternatively $i \rightarrow -i$. $\mathcal{T}$ is anti-unitary: unitary, but also complex conjugates. It acts as 
\begin{equation}
    \mathcal{T}\ket{0} = \ket{0}, \quad \mathcal{T}\ket{\underline{p}} = \ket{-\underline{p}}, \quad \mathcal{T} a^\dagger(\underline{p})\mathcal{T}^\dagger = a^\dagger(-\underline{p}), \quad \mathcal{T}a(\underline{p})\mathcal{T}6 \dagger = a(-\underline{p})
\end{equation}

but
\begin{equation}
    \phi(x) \rightarrow \mathcal{T} \phi(x) \mathcal{T}^\dagger = \phi(-\bar{x}).
\end{equation}

Under \textbf{charge conjugation}, particles and antiparticles are exchanged: $\mathcal{C}\ket{\underline{p},+} = \ket{\underline{p},-}$, $\mathcal{C}\ket{\underline{p},-} = \ket{\underline{p},+}$, so 
\begin{equation}
    \mathcal{C}a^\dagger(\underline{p})\mathcal{C}^\dagger = b^\dagger(\underline{p}), \quad \mathcal{C}b^\dagger(\underline{p})\mathcal{C}^\dagger = a^\dagger(\underline{p}).
\end{equation}

Furthermore (only for complex fields, since scalar fields do not carry any charge),
\begin{equation}
    \mathcal{C}\phi(x)\mathcal{C}^\dagger = \phi^\dagger(x), \quad \mathcal{C}\phi^\dagger(x)\mathcal{C}^\dagger = \phi(x),
\end{equation}
and
\begin{eqnarray}
    \mathcal{C}j^\mu\mathcal{C}^\dagger = -j^\mu, \quad \mathcal{C}Q\mathcal{C}^\dagger = -Q.
\end{eqnarray}

\section{The Dirac Equation}

The Dirac equation, written in covariant form as 
\begin{empheq}[box=\widefbox]{align*}
    (i\gamma^\mu \partial_\mu - m)\psi(x) = 0,
\end{empheq}

contains two objects worth studying in detail: $\gamma^\mu$ and $\psi(x)$. The gamma matrices $\gamma^\mu$ follow the \textit{Clifford algebra}
\begin{equation}
    \left\{\gamma^\mu, \gamma^\nu\right\} = 2\eta^{\mu\nu}
\end{equation}

and have two common representations, known as the Dirac basis and Weyl basis respectively:
\begin{gather}
    \gamma^0 = \begin{pmatrix}\imat{2}\end{pmatrix}, \quad \gamma^i = \begin{pmatrix}0 & \sigma^i \\ -\sigma^i & 0\end{pmatrix}, \quad \gamma^5 = \begin{pmatrix}0 & 1 \\ 1 & 0 \end{pmatrix}, \\
    \gamma^0 = \begin{pmatrix}0 & 1 \\ 1 & 0\end{pmatrix}, \quad \gamma^i = \begin{pmatrix}0 & \sigma^i \\ -\sigma^i & 0\end{pmatrix}, \quad \gamma^5 = \begin{pmatrix}-1 & 0 \\ 0 & 1\end{pmatrix},
\end{gather}

where we have used 
\begin{equation}
    \gamma^5 = i\gamma^0\gamma^1\gamma^2\gamma^3.
\end{equation}

Recalling our knowledge of the Lorentz group from Symmetries of Particles and Fields, Lorentz transformations may be written as
\begin{equation}
    S(\Lambda)=\exp\left(-\frac{i}{4}\omega_{\mu\nu} \sigma^{\mu\nu}\right), \quad \sigma^{\mu\nu} \equiv \frac{i}{2}\left[\gamma^\mu, \gamma^\nu\right].
\end{equation}

Under Lorentz transformations, the four-component object $\psi(x)$ transforms as 
\begin{equation}
    \psi(x) \rightarrow \psi'(x') = S(\Lambda)\psi(x),
\end{equation}

$\gamma^\mu$ transforms as 
\begin{equation}
    \Lambda^\mu_{\, \nu} \gamma^\nu = S(\Lambda)^{-1}\gamma^\mu S(\Lambda).
\end{equation}

We also introduce the conjugate $\bar{\psi} = \psi^\dagger(x)\gamma^0$.

From this, we can construct the 16 Dirac bilinears

\begin{equation}
    \begin{array}{ccc}
    \bar{\psi} \psi & \text { scalar } & 1 \\
    \bar{\psi} \gamma^\mu \psi & \text { vector } & 4 \\
    \bar{\psi} \sigma^{\mu \nu} \psi & \text { tensor } & 6 \\
    \bar{\psi} \gamma^5 \gamma^\mu \psi & \text { axial vector } & 4 \\
    \bar{\psi} \gamma^5 \psi & \text { pseudoscalar } & 1.
    \end{array}
\end{equation}

The Lagrangian for the Dirac field is given by 
\begin{empheq}[box=\widefbox]{align*}
    \mathcal{L} = \bar{\psi}(x)\left(i\slashed{\partial}-m   \right)\psi(x)
\end{empheq}

if we treat $\psi(x)$ and $\bar{\psi}(x)$ as independent fields. Varying the Lagrangian with respect to $\bar{\psi}(x)$ gives the Dirac equation, while varying the Lagrangian with respect to $\psi(x)$ gives $\left(i\partial_\mu \gamma^\mu + m\right)\bar{\psi}(x) = 0$. Just as with complex scalar fields, we have a conserved current and charge 
\begin{equation}
    j^\mu = \bar{\psi}\gamma^\mu\psi, \quad Q = \int d^3x \psi^\dagger\psi.
\end{equation}

The solutions to the Dirac equation are plane wave solutions, the positive and negative energy ones given by
\begin{equation}
    \psi^+(x) = \exp(-ip_\mu x^\mu)\,u(\underline{p}), \quad \psi^-(x) = \exp(ip_\mu x^\mu) v(\underline{p})
\end{equation}

respectively, where $u$ and $v$ are four-component \textit{spinors} also satisfying the Dirac equation:
\begin{gather}
    u(\underline{p}, s) = \sqrt{E+m} \begin{pmatrix}
        \phi^s \\ \frac{\underline{\sigma} \cdot \underline{p}}{E+m} \phi^s
    \end{pmatrix} \\
    v(\underline{p}, s) = \sqrt{E+m} \begin{pmatrix}
        \frac{\underline{\sigma} \cdot \underline{p}}{E+m} \chi^s \\ \chi^s
    \end{pmatrix},
\end{gather}

where 
\begin{gather}
    \phi^1 = \begin{pmatrix}
        1 \\ 0
    \end{pmatrix}, \quad \phi^2 = \begin{pmatrix}
        0 \\ 1
    \end{pmatrix}, \\
    \chi^s = \begin{pmatrix}
        0 \\ 1
    \end{pmatrix}, \quad \chi^2 = \begin{pmatrix}
        -1 \\ 0
    \end{pmatrix}.
\end{gather}

The energy states $s=1,\, 2$ are degenerate in energy; the \textit{helicity} operator
\begin{equation}
    \hat{h}(\underline{p}) = \frac{\underline{\Sigma} \cdot \underline{p}}{|\underline{p}|}, \quad \Sigma \equiv \begin{pmatrix}
        \underline{\sigma} & 0 \\ 0 & \underline{\sigma}
    \end{pmatrix} = -\frac{1}{4} \epsilon_{ijk} [\gamma_j, \gamma_k]
\end{equation}

commutes with the Hamiltonian and measures the spin projected along the direction of motion. In accordance with our earlier definition of $\bar{\psi}(x)$, let us list out a few properties of $u$, $\bar{u} \equiv u^\dagger \gamma^0$, $v$, and $\bar{v} \equiv v^\dagger \gamma^0$:
\begin{gather}
    \sum_s u(p, s) \bar{u} (p, s) = \slashed{p} + m \equiv 2m\Lambda_+ \\
    \sum_s v(p, s) \bar{v} (p, s) = \slashed{p} - m \equiv -2m \Lambda_- \\
    \psi = \alpha u + \beta v \rightarrow \Lambda_+ \psi = \alpha u, \, \Lambda_- \psi = \beta v \\
    \Lambda_+^2 = \Lambda_+, \quad \Lambda_-^2 = \Lambda_-, \Lambda_+ \Lambda_- = 0.
\end{gather}


\end{document}